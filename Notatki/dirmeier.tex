\documentclass[12pt]{amsart}
\usepackage{amssymb} % ten pakiet jest nieodzowny m.in. po to aby zrenderowac instrukcje varsubsetneq
\usepackage{amsmath}
\usepackage{latexsym}
\usepackage{amsfonts}
\usepackage{enumerate}
%\usepackage[mathscr]{eucal}
\usepackage[mathscr]{euscript}
\usepackage{eqlist}
\usepackage{amsthm}
\usepackage{mathtools}
\usepackage{enumitem} % po to aby mozna bylo tworzyc otoczenie
% enumerate z innymi sposobami numeracji itemow
\usepackage{hyperref} % aby moc tworzyc linki w tekscie
%%%%%%%%%%%%%%% Polish letter packages %%%%%%%%%%%%%%%%%%%%%%%%%%%%%%%%
\usepackage[polish]{babel}
\usepackage[utf8]{inputenc}
\usepackage{t1enc}
%%% pakiet do generowania belkotu (wypelniacza):
\usepackage{lipsum}
%%% pakiet do dodawania fikusnych notek `a la ``todo''
\usepackage{todonotes}
%Ten pakiet niby jest m.in. po to aby mozna bylo ,,lamac'' linie w komorkach i ,,glowie'' tabeli
% (otoczenie table)
\usepackage{makecell}
%ten pakiet sluzy do eleganckiego formatowania listingow (programow? lecz zapewne nie tylko)
\usepackage{listings}
%ten pakiet sluzy do sprawdzania m.in. czy odpowiednie pozycje bibliograficzne
%sa faktycznie uzywane w tresci pliku. Przypuszczalnie nalezy go
%uruchamiac tylko podczas weryfikacji pliku - w oficjalnej wersji powininen byc wylaczony.
\usepackage{refcheck}
%pakiet dolaczony na razie po to aby moc utworzyc ,,monstrualne'' kwantyfikatory bigforall i bigexists
\usepackage{graphicx}
%pakiet rotating służy do obracania obiektów, takich jak tekst, tabele, obrazy, wykresy itp.
\usepackage{rotating}
%pakiet do zrecznego tworzenia wielowierszowych tabel.
\usepackage{multirow}
% pierwszy pakiet daje możliwość użycia fontów caligraficznych,
% natiomiast nie mam pojęcia w jakim celu jest ten drugi pakiet,
% ale on jest z całą pewnością powiązany z pierwszym.
\usepackage{calligra}
\usepackage[T1]{fontenc}

\begin{document}
Metryka Alexandra Dirmeiera:
Oryginalna definicja: 
\[ \| n \|_q = \sum_{\substack{\neg k | n \\ k \in \mathbb{N}}} \frac{1}{q^k} \]

\[\rho(n,m) = \| n - m \|_q\]

Oznaczenie:
$\{a \cdot \mathbb{Z} + b\} = \{ a \cdot k + b: k \in \mathbb{Z}\}$.

Teraz bierzemy zbiezny szereg o wyrazach nieujemnych:
$\alpha_k > 0$, $\sum_{k = 1}^\infty \alpha_k < \infty$.

Sprawdzenie że funkcja 
\[
\| n \|_\alpha = \sum_{\substack{\neg k | n \\ k \in \mathbb{N}}} \alpha_k 
\]  
jest faktycznie pseudonormą, czyli że:
\begin{enumerate}
\item
  $\| n \|_\alpha = 0 \iff n = 0;$
\item
  $\|n + m\|_\alpha \leq \| n \|_\alpha + \| m \|_\alpha$. 
\end{enumerate}
jest identyczne i też łatwe jak w przypadku oryginalnego
szeregu geometrycznego używanego przez Dirmeiera.

Sprawdzenie ze $\rho_\alpha$ generuje topologię Furstenberga:

Jesli mamy dana kule: 
$K_\alpha(m,\varepsilon) = \{ n \in \mathbb{Z}: \| n - m\|_\alpha < \varepsilon\}$
to wybieramy naturalne $K$ takie, ze
$\sum_{k = K + 1}^\infty \alpha_k < \varepsilon$ i wtedy zachodzi:
  $\{ K! \cdot \mathbb{Z} + m\} \subseteq K_\alpha(m,\varepsilon)$, 
bowiem 
  $\| K! \cdot l + m - m \|_\alpha = \| K! \cdot l \|_\alpha \leq \sum_{k = K + 1}^\infty < \varepsilon$.
  
Na odwrót, mamy 
$K_\alpha(m, \alpha_n) \subseteq \{n \cdot\mathbb{Z} + m\}$.
Istotnie, niech $M$ całkowite bedzie takie ze 
$\|M - m\|_\alpha < \alpha_n$. Wowczas
$n | M - m$, zatem $M - m = r\cdot n$, czyli $M = r\cdot n + m \in 
\{n \cdot\mathbb{Z} + m\}$. 


Coś bardziej jeszcze ciekawego znalazłem i to znów w pracy Aleksandra Dirmeiera:
Okazuje się że metryka zdefiniowana w następujący sposób również generuje topologię Furstenberga:

\[
\rho(n,m) = \| n - m \|
\]
gdzie
\[
  \| n \| = \frac{1}{M(n)},
\]
przy czym dodatkowo zakładamy że $\|0\| = 0$
a
\[
M(n) = \max(k \geq 1 \colon 1 | n, 2|n, \ldots k | n) = \max(k \geq 1\colon [1:2:\ldots:k] | n).
\]
($[a:b]$ to oczywiście NWW).
Tak naprawdę zamiast $\frac{1}{x}$ można tu wziąć dowolną malejącą i zbieżną do zera
funkcję, to zawsze będzie metryka. 
Dowcip polega na tym że ona znów generuje topologię Furstenberga! Zatem mamy kolejny
przykład metryki ,,Furstenberga''. Wydaje mi się że to ona bardziej 
będzie adekwatna do zbadania w niej zbiorów porowatych. 

Swoją drogą ma ona ciekawą interpretację:
na osi $\mathbb{Z}$ zaznaczamy po kolei: wielokrotności $1$, wielokrotności $2$, itd.itd.
Części wspólne tych wielokrotności tworzą oczywiście sekwencję malejącą 
zbiorów: $W_1, W_2, \ldots$ o przekroju postaci $\{0\}$. 
No to patrzymy na ,,pierścienie'': $W_1 \setminus W_2$, $W_2 \setminus W_3$ itd.
I na każdym takim coraz dalszym pierścieniu kładziemy coś malejącego do zera,
no na przykład na pierwszym $\frac{1}{1}$, na drugim $\frac{1}{2}$, etc.
i wtedy mamy już tę ,,normę'' $\|n\|$, która de facto jest wartością metryki $n$ od zera. 
Dla zmierzenia odległości od $n$ do $m$ wystarczy zmierzyć odległość $n-m$ od zera.

\end{document}
