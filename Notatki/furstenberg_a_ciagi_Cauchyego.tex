\documentclass[12pt]{amsart}
\usepackage{amssymb} % ten pakiet jest nieodzowny m.in. po to aby zrenderowac instrukcje varsubsetneq
\usepackage{amsmath}
\usepackage{latexsym}
\usepackage{amsfonts}
\usepackage{enumerate}
%\usepackage[mathscr]{eucal}
\usepackage[mathscr]{euscript}
\usepackage{eqlist}
\usepackage{amsthm}
\usepackage{mathtools}
\usepackage{enumitem} % po to aby mozna bylo tworzyc otoczenie
% enumerate z innymi sposobami numeracji itemow
\usepackage{hyperref} % aby moc tworzyc linki w tekscie
%%%%%%%%%%%%%%% Polish letter packages %%%%%%%%%%%%%%%%%%%%%%%%%%%%%%%%
\usepackage[polish]{babel}
\usepackage[utf8]{inputenc}
\usepackage{t1enc}
%%% pakiet do generowania belkotu (wypelniacza):
\usepackage{lipsum}
%%% pakiet do dodawania fikusnych notek `a la ``todo''
\usepackage{todonotes}
%Ten pakiet niby jest m.in. po to aby mozna bylo ,,lamac'' linie w komorkach i ,,glowie'' tabeli
% (otoczenie table)
\usepackage{makecell}
%ten pakiet sluzy do eleganckiego formatowania listingow (programow? lecz zapewne nie tylko)
\usepackage{listings}
%ten pakiet sluzy do sprawdzania m.in. czy odpowiednie pozycje bibliograficzne
%sa faktycznie uzywane w tresci pliku. Przypuszczalnie nalezy go
%uruchamiac tylko podczas weryfikacji pliku - w oficjalnej wersji powininen byc wylaczony.
%\usepackage{refcheck}
%pakiet dolaczony na razie po to aby moc utworzyc ,,monstrualne'' kwantyfikatory bigforall i bigexists
\usepackage{graphicx}
%pakiet rotating służy do obracania obiektów, takich jak tekst, tabele, obrazy, wykresy itp.
\usepackage{rotating}
%pakiet do zrecznego tworzenia wielowierszowych tabel.
\usepackage{multirow}
% pierwszy pakiet daje możliwość użycia fontów caligraficznych,
% natiomiast nie mam pojęcia w jakim celu jest ten drugi pakiet,
% ale on jest z całą pewnością powiązany z pierwszym.
\usepackage{calligra}
\usepackage[T1]{fontenc}


%---------------------------------------------------------------------
% Theorems 
%---------------------------------------------------------------------
 
% Theorem style 'plain' are for: Theorem, Lemma, Corollary,
% Proposition, Conjecture, Criterion, Algorithm
\theoremstyle{plain}
\newtheorem{theorem}{Theorem}[section]
\newtheorem{lemma}[theorem]{Lemma}
\newtheorem{corollary}[theorem]{Corollary}
\newtheorem{conclusion}[theorem]{Corollary}
\newtheorem{claim}[theorem]{Claim}
\newtheorem{fact}[theorem]{Fact}
\newtheorem{proposition}[theorem]{Proposition}
\newtheorem{axiom}{Axiom}
% Theorem style 'definition' are for: Definition, Condition, Problem,
% Example
\theoremstyle{definition}
\newtheorem{definition}[theorem]{Definition}
\newtheorem{example}[theorem]{Example}
\newtheorem{exercise}{Exercise}
\newtheorem*{solution}{Solution}
\newtheorem{remark}[theorem]{Remark}
\newtheorem{Problem}[theorem]{Problem}
% Theorem style 'remark' are for: Remark, Note, Notation, Claim,
% Summary, Acknowledgement, Case, Conclusion
\theoremstyle{remark}
\newtheorem*{notation}{Notation}
\newtheorem*{acknowledgement}{Acknowledgement}


%%%%%% makra:

%A
\newcommand{\afc}{\mathrm{AFC}}
\newcommand{\afcbar}{\overline{\afc}}
\newcommand{\arr}{\rightarrow}
\newcommand{\Arr}{\Rightarrow}

%B
\newcommand{\baire}{\omega^{\omega}}
\newcommand{\Bor}{\mbox{$\mathcal{Bor}$}}

%C
\newcommand{\ca}{2^{\omega}}
\newcommand{\cantor}{\ca}
\newcommand{\Card}[1]{\Vert #1 \Vert}

%D
\newcommand{\dom}{{\rm dom}}
\newcommand{\dummy}{{\tt Blah blah blah}}

%E
\newcommand{\Even}{\hbox{\rm \tiny Even}}

%F
\newcommand{\finsub}{[\omega]^{<\omega}}
\newcommand{\forces}{\mathrel{\|}\joinrel\mathrel{-}}

%G
\newcommand{\Graph}{\hbox{\it Graph}}
\newcommand{\Graphit}{{\mathit Graph}}
\newcommand{\Graphh}{\operatorname{Graph}}

%H
\newcommand{\homeomorphic}{\approx}

%I
\newcommand{\incr}{\omega^{\uparrow \omega }}
\newcommand{\infsub}{[\omega]^{\omega}}
\newcommand{\integers}{\mathbb{Z}}

%L
\newcommand{\la}{\langle}

%M
\newcommand{\meager}{{\cal MGR}}
\newcommand{\minideal}{${\cal F}_{\hbox{\rm \scriptsize min}}(\neg
  D)\;$}

%N
\newcommand{\neglig}{{\cal N}}
\newcommand{\nnatural}{\mathbb{N}}

%O
\newcommand{\Odd}{\hbox{\rm \tiny Odd}}

%P
\newcommand{\Part}{{\it Part}}
\newcommand{\Perf}{{\it Perf}}
%%%\newcommand{\proof}{\flushleft{ \sc Proof. } \\ }
\newcommand{\Proof}[1]{\bigbreak\noindent{\bf Proof #1}\enspace}

%Q
%%%\newcommand{\qed}{{\hfill\vrule height6pt width6pt depth1pt\medskip}}
%\newcommand{\qed}{\sharp}
% to poniżej w komentarz o ile używamy pakietu stix
%\newcommand{\QED}{\hspace{0.1in} \Box \vspace{0.1in}}

%R
\newcommand{\ra}{\rangle}
\newcommand{\ran}{{\rm ran}}
\newcommand{\rational}{\mathbb{Q}}
\newcommand{\real}{\mathbb{R}}

%S
\newcommand{\seq}{\subseteq}
%%%\newcommand{\square}{\hbox{\ \ \ \ \ \vrule\vbox{\hrule\phantom{o}\hrule}\vrule}}
% a small restriction:
\newcommand{\srestriction}{{\hbox{${\scriptstyle\,|\grave{}\,}$}}}
\newcommand{\supp}{\mathit{supp}}

%U
\newcommand{\up}{\uparrow}
\newcommand{\ucrz}{UCR_0}

%%%%%%%%%%%%%%%%%%%%%% Calligraphic font commands %%%%%%%%%%%%%%%%%%%%%%%%%%
\newcommand{\calA}{\mathcal{A}}
\newcommand{\calB}{\mathcal{B}}
\newcommand{\calC}{\mathcal{C}}
\newcommand{\calD}{\mathcal{D}}
\newcommand{\calE}{\mathcal{E}}
\newcommand{\calF}{\mathcal{F}}
\newcommand{\calG}{\mathcal{G}}
\newcommand{\calH}{\mathcal{H}}
\newcommand{\calI}{\mathcal{I}}
\newcommand{\calJ}{\mathcal{J}}
\newcommand{\calK}{\mathcal{K}}
\newcommand{\calL}{\mathcal{L}}
\newcommand{\calM}{\mathcal{M}}
\newcommand{\calN}{\mathcal{N}}
\newcommand{\calO}{\mathcal{O}}
\newcommand{\calP}{\mathcal{P}}
\newcommand{\calQ}{\mathcal{Q}}
\newcommand{\calR}{\mathcal{R}}
\newcommand{\calS}{\mathcal{S}}
\newcommand{\calT}{\mathcal{T}}
\newcommand{\calU}{\mathcal{U}}
\newcommand{\calV}{\mathcal{V}}
\newcommand{\calW}{\mathcal{W}}
\newcommand{\calX}{\mathcal{X}}
\newcommand{\calY}{\mathcal{Y}}
\newcommand{\calZ}{\mathcal{Z}}
%%%%%%%%%%%%%%%%%%%%%% inne %%%%%%%%%%%%%%%%%%%%%%%%%%%%%%%%%%%%%%%
\newcommand{\SqrFr}{\mathbb{SF}}
\newcommand{\Primes}{\mathit{Primes}}
\newcommand{\mathint}{\mathit{int}}
\newcommand{\mathcl}{\mathit{cl}}
\newcommand{\exampleGFnotK}{\mathit{Ex}}
%%%%%%%%%%%%%%%%%%%%%% Some Greek fonts %%%%%%%%%%%%%%%%%%%%%%%%%%%%%%%%%%%%%%%
\newcommand{\oo}{\omega}
\newcommand{\bb}{\beta}
\newcommand{\dd}{\delta}
\newcommand{\ee}{\varepsilon}
\newcommand{\kk}{\kappa}
%%%\newcommand{\th}{\theta}
\newcommand{\stirlingii}{\genfrac{\{}{\}}{0pt}{}}
%%% makra dla utworzenia monstrualnych kwantyfikatorow:
\newcommand{\bigforall}{\mbox{\Large $\mathsurround0pt\forall$}}
\newcommand{\bigexists}{\mbox{\Large $\mathsurround0pt\exists$}}

%%% Makra charakterystyczne dla tej pracy:
\newcommand{\PrimeOdd}{\mathit{Prime}\setminus \{2\}}

\begin{document}
Metryki na $\integers$ które generują topologię Furstenberga:
\begin{itemize}
\item
  $\|n\| = \sum_{\substack{\neg k | n\\ k \in \mathbf{N}}} \alpha_k$
  ($(\alpha_k)$ to ciąg liczb dodatnich i takich że $\sum_{k=1}^\infty \alpha_k < \infty$);
\item
  $\|n\| = 1/\max\{k: [1:\ldots:k] | n\}$;
\item
  $\|n\| = 1/\max\{k: k! | n\}$.
\end{itemize}


\begin{lemma}
  Jeśli $c_1, c_2,\ldots \in \mathbf{Z}$
  są dowolne to wtedy ciąg 
\[
  a_n = \sum_{j=1}^n c_j \cdot j!
\]
jest ciągiem Cauchyego w metrykach Furstenberga (chodzi o dowolną z metryk
w uwadze powyżej).
\end{lemma}

\begin{proof}
Wystarczy tylko zauważyć że dla $m > n$ mamy
$a_m - a_n = \sum_{j=n+1}^m c_j \cdot j!$ a każde takie $j!$ jest 
podzielne przez $n!$, więc $k | a_m - a_n$ dla $k=1,2,\ldots,n$, 
co pozwala dla każdego $\varepsilon > 0$ dobrać tak duże $n$
że $\|a_m - a_n\| < \varepsilon$ dla $m > n$.
Z tego już wynika warunek Cauchy'ego dla ciągu $(a_n)$.
\end{proof}

Niech teraz $K \colon \PrimeOdd \to \mathbb{Z}$ będzie
ustaloną bijekcją. Definiujemy indukcyjnie liczby $c_n$:

\flushleft{Przypadek 1:} $n + 1 \not\in \PrimeOdd$.
Wówczas za $c_n$ kładziemy dowolną liczbę całkowitą,
może to być nawet i zero.

\flushleft{Przypadek 2:} $n + 1 \in \PrimeOdd$.
Wybieramy $c_n\in\integers$ tak aby
\begin{equation}\label{warunek-z-przypadku-2}\tag*{(*)}
c_n \not\equiv \sum_{j = 1}^{n - 1} c_j \cdot j! - K(n + 1) \pmod{n + 1}.
\end{equation}
W ten sposób konstrukcja indukcyjna liczb $c_n$ jest zakończona.
Zatem mamy skonstruowany ciąg $(c_n)$ a dla niego definiujemy,
jak opisano to wyżej, ciąg $(a_n)$.
\begin{fact}\label{fact:fakt-nr-1}
  Dla $p \in \PrimeOdd$ mamy
  $a_{p-1} - K(p) \not\equiv 0 \pmod p$.
\end{fact}
\begin{proof}
  Mamy:
  \[a_{p-1} - K(p) = \sum_{j = 1}^{p - 1} c_j \cdot j!\]
  \[= \sum_{j = 1}^{p - 2} c_j \cdot j! + c_{p - 1} (p - 1)! - K(p)\]
  Ponieważ $p$ jest liczbą pierwszą, zatem wiemy z
  twierdzenia Wilsona że $(p-1)! \equiv -1\pmod p$.
  Mamy zatem dalej:
  \[\sum_{j = 1}^{p - 1} c_j \cdot j! + c_{p - 1} (p - 1)! - K(p) =\]
  \[\sum_{j = 1}^{p - 2} c_j \cdot j! - c_{p - 1} - K(p) \not\equiv 0 \pmod p.\]
  (po zastosowaniu warunku \ref{warunek-z-przypadku-2}
\end{proof}

\begin{fact}\label{fact:fakt-nr-2}
  Dla $n \geq p$, $p \in \PrimeOdd$ mamy że
  $a_n - K(p) \not\equiv 0 \pmod p$.
\end{fact}
\begin{proof}
  Mamy:
  \[a_n - K(p) = \sum_{j=1}^{n} c_j \cdot j! - K(p) = \]
  \[\sum_{j=1}^{p-1} c_j \cdot j! - K(p) + \sum_{j=p}^{n} c_j \cdot j!. \]

  Z Faktu \ref{fact:fakt-nr-1} wiemy że
  $a_{p-1} - K(p) \not\equiv 0 \pmod p$,
  czyli $\neg (p | a_{p-1} - K(p))$, a znaczy to że
  $\neg (p |\sum_{j=1}^{p-1} c_j \cdot j! - K(p))$.
  Dla $j = p, \ldots, n$ mamy że $p | j!$, czyli
  $\sum_{j=p}^{n} c_j\cdot j!$ jest podzielna przez $p$.

  Finalnie zatem
  $\sum_{j=1}^{p-1} c_j \cdot j! - K(p) +  \sum_{j=p}^{n} c_j \cdot j!$
  nie dzieli się przez $p$, czyli 
  $a_n - K(p)$ nie dzieli się przez $p$. 
\end{proof}

Dla każdego $k\in\mathbb{Z}$ wybieramy $p(k)\in\PrimeOdd$ takie że
$K(p(k)) = k$ i wtedy, na mocy Faktu \ref{fact:fakt-nr-2}
dla $n > p$, $a_n - k$ nie dzieli się przez p(k).
Stąd $\|a_n - k\| \geq \varepsilon$ (np dla 
$\|\cdot \|_q$ mamy $\varepsilon = 1/q^p$) dla $n > p$, czyli ciąg $(a_n)$
nie może być zbieżny do $k$ w metryce $\|\cdot\|$.
\end{document}
