\documentclass[12pt]{amsart}
\usepackage{amssymb} % ten pakiet jest nieodzowny m.in. po to aby zrenderowac instrukcje varsubsetneq
\usepackage{amsmath}
\usepackage{latexsym}
\usepackage{amsfonts}
\usepackage{enumerate}
%\usepackage[mathscr]{eucal}
\usepackage[mathscr]{euscript}
\usepackage{eqlist}
\usepackage{amsthm}
\usepackage{mathtools}
\usepackage{enumitem} % po to aby mozna bylo tworzyc otoczenie
% enumerate z innymi sposobami numeracji itemow
\usepackage{hyperref} % aby moc tworzyc linki w tekscie
%%%%%%%%%%%%%%% Polish letter packages %%%%%%%%%%%%%%%%%%%%%%%%%%%%%%%%
\usepackage[polish]{babel}
\usepackage[utf8]{inputenc}
\usepackage{t1enc}
%%% pakiet do generowania belkotu (wypelniacza):
\usepackage{lipsum}
%%% pakiet do dodawania fikusnych notek `a la ``todo''
\usepackage{todonotes}
%Ten pakiet niby jest m.in. po to aby mozna bylo ,,lamac'' linie w komorkach i ,,glowie'' tabeli
% (otoczenie table)
\usepackage{makecell}
%ten pakiet sluzy do eleganckiego formatowania listingow (programow? lecz zapewne nie tylko)
\usepackage{listings}
%ten pakiet sluzy do sprawdzania m.in. czy odpowiednie pozycje bibliograficzne
%sa faktycznie uzywane w tresci pliku. Przypuszczalnie nalezy go
%uruchamiac tylko podczas weryfikacji pliku - w oficjalnej wersji powininen byc wylaczony.
\usepackage{refcheck}
%pakiet dolaczony na razie po to aby moc utworzyc ,,monstrualne'' kwantyfikatory bigforall i bigexists
\usepackage{graphicx}
%pakiet rotating służy do obracania obiektów, takich jak tekst, tabele, obrazy, wykresy itp.
\usepackage{rotating}
%pakiet do zrecznego tworzenia wielowierszowych tabel.
\usepackage{multirow}
% pierwszy pakiet daje możliwość użycia fontów caligraficznych,
% natiomiast nie mam pojęcia w jakim celu jest ten drugi pakiet,
% ale on jest z całą pewnością powiązany z pierwszym.
\usepackage{calligra}
\usepackage[T1]{fontenc}

\begin{document}
{\bf Fakt:} Jeśli $c_1, c_2,\ldots \in \mathbf{Z}$
to wtedy ciąg 
\[
  a_n = \sum_{j=1}^n c_j \cdot j!
\]
jest ciągiem Cauchyego w metrykach Furstenberga (chodzi o każdą z metryk: 
$\|n\| = \sum_{\substack{\neg k | n\\ k \in \mathbf{N}}} \alpha_n$, lub 
$\|n\| = 1/\max\{k: [1:\ldots:k] | n\}$.
).

Wystarczy tylko zauważyć że dla $m > n$ mamy
$a_m - a_n = \sum_{j=n+1}^m c_j \cdot j!$ a każde takie $j!$ jest 
podzielne przez $n!$, więc $k | a_m - a_n$ dla $k=1,2,\ldots,n$, 
co pozwala dla każdego $\varepsilon > 0$ dobrać tak duże $n$
że $\|a_m - a_n\| < \varepsilon$ dla $m > n$.


Niech teraz $K \colon \mathit{Prime}\setminus \{2\} \to \mathbb{Z}$ będzie
ustaloną bijekcją. Definiujemy indukcyjnie liczby $c_n$:

Dla $p \in \mathit{Prime}\setminus \{2\}$
wiemy z twierdzenia Wilsona że $(p-1)! \equiv -1(\pmod p)$.
Wybieramy więc dowolne $c_{p-1}$ ale takie by
$c_{p-1} \not\equiv \sum_{j=1}^{p-2} c_j \cdot j! - K(p) (\pmod p)$.

Dla pozostałych $n$ kładziemy $c_n$ dowolnie (możemy np. nawet zero).

W ten sposób konstruujemy $(c_n)$. I definiujemy j.w. ciąg $(a_n)$.

Mamy:
$a_{p-1} - K(p) \not\equiv 0 (\pmod p)$, 
i dla $n > p$: skoro $\sum_{j=p}^{n} c_j\cdot j!$ jest podzielna przez $p$,
więc $a_n - K(p)$ nie dzieli się przez $p$. 

Dla każdego $k\in\mathbb{Z}$ wybieramy $p\in\mathit{Prime}\setminus \{2\}$ takie że
$K(p) = k$ i wtedy dla $n > p$, $a_n - k$ nie dzieli się przez p,
stąd $\|a_n - k\| \geq \varepsilon$ (np dla 
$\|\cdot \|_q$ mamy $\varepsilon = 1/q^p$) dla $n > p$, czyli $(a_n)$
nie może być zbieżny do $k$.
\end{document}
